\documentclass[english,onecolumn]{IEEEtran}
\usepackage[T1]{fontenc}
\usepackage[colorlinks]{hyperref}
\usepackage{color,xcolor}
\usepackage{amsthm,amssymb,amsfonts,amsmath}
\usepackage{mathtools}
\usepackage{algorithm}
\usepackage{algpseudocode}
% for matlab code highlight
% load package with ``framed'' and ``numbered'' option.
\usepackage[framed,numbered,autolinebreaks,useliterate]{mcode}

\providecommand{\U}[1]{\protect\rule{.1in}{.1in}}
\topmargin            -18.0mm
\textheight           226.0mm
\oddsidemargin      -4.0mm
\textwidth            166.0mm
\def\baselinestretch{1.5}




% math operator macros
\newcommand{\madj}{\mathop{\mathrm{adj}}}
\newcommand{\trace}{\mathop{\mathrm{tr}}}


\DeclarePairedDelimiter{\abs}{\lvert}{\rvert}
\DeclarePairedDelimiter{\norm}{\lVert}{\rVert} 
\DeclarePairedDelimiter{\pnorm}{\lVert}{\rVert_p}



% special letters
\newcommand{\x}{{\mathbf{x}}} % vector x
\newcommand{\y}{{\mathbf{y}}} % vector y
\newcommand{\z}{{\mathbf{z}}} % vector z
\newcommand{\0}{{\mathbf{0}}} % zero vector

\newcommand{\A}{{\mathbf{A}}} % matrix A
\newcommand{\B}{{\mathbf{B}}} % matrix B
\newcommand{\matC}{{\mathbf{C}}} % matrix C
\newcommand{\matD}{{\mathbf{D}}} % matrix D
\newcommand{\matE}{{\mathbf{E}}} % matrix E
\newcommand{\matH}{{\mathbf{H}}} % matrix H
\newcommand{\matK}{{\mathbf{K}}} % matrix K
\newcommand{\X}{{\mathbf{X}}} % matrix X
\newcommand{\Y}{{\mathbf{Y}}} % matrix Y
\newcommand{\Z}{{\mathbf{Z}}} % matrix Z
\newcommand{\M}{{\mathbf{M}}} % matrix M
\newcommand{\I}{{\mathbf{I}}} % identity matrix

\newcommand{\calV}{{\mathcal{V}}} % subspace V
\newcommand{\calU}{{\mathcal{U}}} % subspace U
\newcommand{\calS}{{\mathcal{S}}} % subspace S
\newcommand{\calM}{{\mathcal{M}}} % subspace M
\newcommand{\calN}{{\mathcal{N}}} % subspace N
\newcommand{\calR}{\mathop{\mathcal{R}}} % range
\renewcommand{\calN}{\mathop{\mathcal{N}}} % nullspace

% number fields
\newcommand{\R}{\mathbb{R}} % real numbers
\newcommand{\C}{\mathbb{C}} % complex numbers

\newcommand{\bigO}{{\mathcal{O}}} % big-Oh notation









\begin{document}

\begin{center}
	\textbf{\LARGE{SI231B: Matrix Computations, 2025 Fall}}\\
	{\Large Homework Set \#3}\\
	% \texttt{Prof. Jie Lu}
\par\end{center}

\noindent
\rule{\linewidth}{0.4pt}
% \noindent
% \rule{\linewidth}{0.4pt}
{\bf Acknowledgements:}
\begin{enumerate}
	\item Deadline: {\bf \textcolor{red}{2025-11-29 23:59:59}}
    \item Please submit the PDF file to \hyperlink{https://www.gradescope.com/}{gradescope}. Course entry code: N2382J.
    \item You have 5 ``free days'' in total for all late homework submissions.
    \item If your homework is handwritten, please make it clear and legible.
    \item All your answers are required to be in English. 
    \item Write down the major steps for deriving the solution; otherwise you may loss points.
\end{enumerate}
\rule{\linewidth}{0.4pt}


% =======================================================================================
\newpage
\noindent \textbf{Problem 1. (Eigenvalue)} (\textcolor{blue}{15 points})

\begin{enumerate}
Let \( \mathbf{A} \in \mathbb{R}^{3 \times 3} \) be a real symmetric matrix satisfying the matrix equation \( \mathbf{A}^3 - 2\mathbf{A}^2 - \mathbf{A} + 2\mathbf{I} = \mathbf{0} \).


\item Find all eigenvalues of matrix \( \mathbf{A} \).
(\textcolor{blue}{10 points})

\item Compute \( \det(\mathbf{A} + \mathbf{I}) \).(\textcolor{blue}{5 points})


\end{enumerate}

















\bigskip

\noindent{\bf Solution:}











% =======================================================================================
\newpage
\noindent \textbf{Problem 2. (Eigenvector and similarity)} (\textcolor{blue}{15 points}) 


\begin{enumerate}
\item Let \( \mathbf{A} \in \mathbb{R}^{n \times n} \) have two distinct eigenvalues \( \lambda_1 \) and \( \lambda_2 \) with corresponding eigenvectors \( \boldsymbol{v}_1 \) and \( \boldsymbol{v}_2 \) respectively.\\
Prove that: If \( k_1\boldsymbol{v}_1 + k_2\boldsymbol{v}_2 \) (where the scalars \( k_1, k_2 \) are not both zero) is an eigenvector of \( \mathbf{A} \), then \( k_1k_2 = 0 \).
  (\textcolor{blue}{7 points}) 
\item Suppose two matrices have the same characteristic polynomial,
determinant, rank, nullity, trace, eigenvalues, algebraic multiplicity, geometric multiplicity. Are they similar? Justify your answer.
(\textcolor{blue}{8 points})

\end{enumerate}

\bigskip

\noindent \textbf{Solution:}




% =======================================================================================
\newpage
\noindent \textbf{Problem 3. (Diagonalization)} \hfill (\textcolor{blue}{20 points}) 

\begin{enumerate}
\item Determine if each of the following matrices is diagonalizable.
\[
\mathbf{A} = \begin{bmatrix}
    -2 & 1 & 1 \\ 
    0 & 2 & 0 \\ 
    -4 & 1 & 3 
\end{bmatrix}, \quad
\mathbf{B} = \begin{bmatrix}
    2 & 4 & 0 \\ 
    0 & 2 & 0 \\ 
    0 & 5 & 2 
\end{bmatrix}
\]
\textcolor{blue}{(10 points)}

\item If it is diagonalizable, diagonalize the matrix using a similarity transformation. \textcolor{blue}{(5 points)}

\item Compute \( \mathbf{A}^{100} \). \textcolor{blue}{(5 points)}
\end{enumerate}

\bigskip

\noindent \textbf{Solution:}









% =======================================================================================
\newpage
\noindent \textbf{Problem 4. (Schur Decomposition) } (\textcolor{blue}{20 points})

\begin{enumerate}
    \item Let 
    \[
    A = \begin{bmatrix} 
    0 & 1 & 0 \\ 
    0 & 0 & 1 \\ 
    6 & -11 & 6 
    \end{bmatrix}.
    \]
    Compute the real Schur decomposition \( A = U T U^T \), where \( U \) is a real orthogonal matrix and \( T \) is upper triangular. \textbf{The results are reported rounded to four decimal places.} (\textcolor{blue}{12 points})
    \item Suppose \( A \in \mathbb{C}^{n \times n} \) has $n$ distinct eigenvalues. Show that if \( Q^{H} A Q = T \) is its Schur decomposition and \( AB = BA \), then \( Q^{H} B Q \) is upper triangular. (Hint: let \( T \in \mathbb{C}^{n \times n} \) be an upper triangular matrix whose diagonal entries \( t_{11}, \dots, t_{nn} \) are distinct. If \( C \in \mathbb{C}^{n \times n} \) satisfies \( TC = CT \), then \( C \) must be upper triangular.) (\textcolor{blue}{8 points})
\end{enumerate}




\bigskip


\noindent \textbf{Solution:}







% =======================================================================================
\newpage
\noindent \textbf{Problem 5. (Variational Characterizations)} (\textcolor{blue}{10 points})

For  matrix $\A\in\mathbb{R}^{r\times r}$, prove $||\A||_2=\max_{||\z||_2=1} ||\A\z||_2=\max_{||\z||_2=||\y||_2=1} |\y^T\A\z|$. 


\noindent{\bf Solution:}


   

% =======================================================================================
\newpage
\noindent \textbf{Problem 6. (Power Iteration)} \hfill (\textcolor{blue}{20 points})

Consider 
$$A=\begin{bmatrix}
    2&1\\1&2
\end{bmatrix},B=\begin{bmatrix}
    2&3 \\ 1&4
\end{bmatrix}.$$
\begin{enumerate}
    \item Apply the power method with infinite-norm normalization to each matrix (Compute only four steps), starting from the initial vector 
$\mathbf{x}^{(0)} = \begin{bmatrix} -1 & 2 \end{bmatrix}^T$. Note: To implement the infinite-norm normalization, replace the normalization step $\mathbf{v}^{(k)} = \frac{\tilde{\mathbf{v}}^{(k)}}{\|\tilde{\mathbf{v}}^{(k)}\|_2}$ used in the slides with $\mathbf{v}^{(k)} = \frac{\tilde{\mathbf{v}}^{(k)}}{\|\tilde{\mathbf{v}}^{(k)}\|_\infty}$, where $\|\cdot\|_\infty$ denotes the infinity norm. $\mathbf{v}^{(k)}$ approximates a dominant eigenvector at iteration $k$ and $\lambda^{(k)} = (A\mathbf{v}^{(k)})^T\mathbf{v}^{(k)}/||\mathbf{v}^{(k)}||_2^2$. \textbf{The results are reported rounded to four decimal places.} (\textcolor{blue}{10 points})
    \item Compute the ratios $\lambda_2/\lambda_1$ for $A$ and $B$, where $\lambda_1$ and $\lambda_2$ denote the eigenvalues of largest and second-largest magnitude, respectively. For which matrix do you expect faster convergence of the power method? Explain your reason. (\textcolor{blue}{10 points})
\end{enumerate}
\bigskip

\noindent \textbf{Solution:}



\end{document}
