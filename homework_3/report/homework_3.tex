\documentclass{article}

\title{\Huge Matrix Computation Homework 1}
\author{\normalsize Yifan Zhang  2025251018  zhangyf52025@shanghaitech.edu.cn}

% import peckage
\usepackage{indentfirst}
\usepackage{amssymb}
\usepackage{amsmath}
\usepackage{array}
\usepackage{mathtools}
\usepackage{parskip}

\newcommand{\madj}{\mathop{\mathrm{adj}}}
\newcommand{\trace}{\mathop{\mathrm{tr}}}

\DeclarePairedDelimiter{\abs}{\lvert}{\rvert}
\DeclarePairedDelimiter{\norm}{\lVert}{\rVert}
\DeclarePairedDelimiter{\pnorm}{\lVert}{\rVert_p}

% special letters
\newcommand{\x}{{\mathbf{x}}} % vector x
\newcommand{\y}{{\mathbf{y}}} % vector y
\newcommand{\z}{{\mathbf{z}}} % vector z
\newcommand{\0}{{\mathbf{0}}} % zero vector

\newcommand{\A}{{\mathbf{A}}} % matrix A
\newcommand{\B}{{\mathbf{B}}} % matrix B
\newcommand{\matC}{{\mathbf{C}}} % matrix C
\newcommand{\matD}{{\mathbf{D}}} % matrix D
\newcommand{\matE}{{\mathbf{E}}} % matrix E
\newcommand{\matH}{{\mathbf{H}}} % matrix H
\newcommand{\matK}{{\mathbf{K}}} % matrix K
\newcommand{\X}{{\mathbf{X}}} % matrix X
\newcommand{\Y}{{\mathbf{Y}}} % matrix Y
\newcommand{\Z}{{\mathbf{Z}}} % matrix Z
\newcommand{\M}{{\mathbf{M}}} % matrix M
\newcommand{\I}{{\mathbf{I}}} % identity matrix

\newcommand{\calV}{{\mathcal{V}}} % subspace V
\newcommand{\calU}{{\mathcal{U}}} % subspace U
\newcommand{\calS}{{\mathcal{S}}} % subspace S
\newcommand{\calM}{{\mathcal{M}}} % subspace M
\newcommand{\calN}{{\mathcal{N}}} % subspace N
\newcommand{\calR}{\mathop{\mathcal{R}}} % range
\renewcommand{\calN}{\mathop{\mathcal{N}}} % nullspace

% number fields
\newcommand{\R}{\mathbb{R}} % real numbers
\newcommand{\C}{\mathbb{C}} % complex numbers

\newcommand{\bigO}{{\mathcal{O}}} % big-Oh notation

\newcommand{\vect}[1]{\mathbf{#1}}

\begin{document}
\maketitle

\section*{Problem 1. (Eigenvalue)}

\subsection*{1)}

\textbf{Solution.}

Let $\lambda$ be an eigenvalue of the matrix $A$,
and let $v$ be the corresponding eigenvector.
Then, we have:
$$ Av = \lambda v $$

Since,
$$
P(A) = A^3 - 2A^2 - A + 2I = 0
$$
any eigenvalue $\lambda$ of $A$ must satisfy the corresponding scalar polynomial equation $P(\lambda) = 0$.

Substituting $\lambda$ into the polynomial:
$$
\lambda^3 - 2\lambda^2 - \lambda + 2 = 0
$$

Then we can get that:
\begin{align*}
\lambda^2(\lambda - 2) - 1(\lambda - 2) &= 0 \\
(\lambda^2 - 1)(\lambda - 2) &= 0 \\
(\lambda - 1)(\lambda + 1)(\lambda - 2) &= 0
\end{align*}

Thus,we can get that
$$
\lambda \in \{1, -1, 2\}
$$

Then, we can get the $\lambda$ pairs:

\begin{table}[h]
\centering
\begin{tabular}{|c|c|c|c|c|c|c|c|c|c|c|}
  \hline
  $\lambda_1$ & -1 & -1 & -1 & -1 & -1 & -1 & 1 & 1 & 1 & 2 \\
  $\lambda_2$ & -1 & -1 & -1 & 1  & 1  & 2  & 1 & 1  & 2  & 2 \\
  $\lambda_3$ & -1 & 1  & 2  & 1  & 2  & 2  & 1 & 2 & 2 & 2 \\
  \hline
\end{tabular}
\caption{$\lambda$ pairs}
\label{tab:basic_example}
\end{table}

\subsection*{2)}

\textbf{Solution.}

Since the eigenvalues of $A$ are:
$$\lambda_1 = 1, \quad \lambda_2 = -1, \quad \lambda_3 \in \{-1, 1, 2\}$$
we can get that the eigenvalues of $(A + I)$ are:
$$\lambda_1' = 1 + 1=2, \quad \lambda_2' = -1 + 1 =0, \quad \lambda_3' \in \{0, 2, 3\}$$

Thus,
$$
det(A + I) = \lambda_1' \cdot \lambda_2' \cdot \lambda_3' = 0
$$

\section*{Problem 2. (Eigenvector and similarity)}

\end{document}
