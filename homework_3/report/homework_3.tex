\documentclass{article}

\title{\Huge Matrix Computation Homework 1}
\author{\normalsize Yifan Zhang  2025251018  zhangyf52025@shanghaitech.edu.cn}

% import peckage
\usepackage{indentfirst}
\usepackage{amssymb}
\usepackage{amsmath}
\usepackage{array}
\usepackage{mathtools}
\usepackage{parskip}

\newcommand{\madj}{\mathop{\mathrm{adj}}}
\newcommand{\trace}{\mathop{\mathrm{tr}}}

\DeclarePairedDelimiter{\abs}{\lvert}{\rvert}
\DeclarePairedDelimiter{\norm}{\lVert}{\rVert}
\DeclarePairedDelimiter{\pnorm}{\lVert}{\rVert_p}

% special letters
\newcommand{\x}{{\mathbf{x}}} % vector x
\newcommand{\y}{{\mathbf{y}}} % vector y
\newcommand{\z}{{\mathbf{z}}} % vector z
\newcommand{\0}{{\mathbf{0}}} % zero vector

\newcommand{\A}{{\mathbf{A}}} % matrix A
\newcommand{\B}{{\mathbf{B}}} % matrix B
\newcommand{\matC}{{\mathbf{C}}} % matrix C
\newcommand{\matD}{{\mathbf{D}}} % matrix D
\newcommand{\matE}{{\mathbf{E}}} % matrix E
\newcommand{\matH}{{\mathbf{H}}} % matrix H
\newcommand{\matK}{{\mathbf{K}}} % matrix K
\newcommand{\X}{{\mathbf{X}}} % matrix X
\newcommand{\Y}{{\mathbf{Y}}} % matrix Y
\newcommand{\Z}{{\mathbf{Z}}} % matrix Z
\newcommand{\M}{{\mathbf{M}}} % matrix M
\newcommand{\I}{{\mathbf{I}}} % identity matrix

\newcommand{\calV}{{\mathcal{V}}} % subspace V
\newcommand{\calU}{{\mathcal{U}}} % subspace U
\newcommand{\calS}{{\mathcal{S}}} % subspace S
\newcommand{\calM}{{\mathcal{M}}} % subspace M
\newcommand{\calN}{{\mathcal{N}}} % subspace N
\newcommand{\calR}{\mathop{\mathcal{R}}} % range
\renewcommand{\calN}{\mathop{\mathcal{N}}} % nullspace

% number fields
\newcommand{\R}{\mathbb{R}} % real numbers
\newcommand{\C}{\mathbb{C}} % complex numbers

\newcommand{\bigO}{{\mathcal{O}}} % big-Oh notation

\newcommand{\vect}[1]{\mathbf{#1}}

\begin{document}
\maketitle

\section*{Problem 1. (Eigenvalue)}

\subsection*{1)}

\textbf{Solution.}

Let $\lambda$ be an eigenvalue of the matrix $A$,
and let $v$ be the corresponding eigenvector.
Then, we have:
$$ Av = \lambda v $$

Since,
$$
P(A) = A^3 - 2A^2 - A + 2I = 0
$$
any eigenvalue $\lambda$ of $A$ must satisfy the corresponding scalar polynomial equation $P(\lambda) = 0$.

Substituting $\lambda$ into the polynomial:
$$
\lambda^3 - 2\lambda^2 - \lambda + 2 = 0
$$

Then we can get that:
\begin{align*}
\lambda^2(\lambda - 2) - 1(\lambda - 2) &= 0 \\
(\lambda^2 - 1)(\lambda - 2) &= 0 \\
(\lambda - 1)(\lambda + 1)(\lambda - 2) &= 0
\end{align*}

Thus,we can get that
$$
\lambda \in \{1, -1, 2\}
$$

Then, we can get the $\lambda$ pairs:

\begin{table}[h]
\centering
\begin{tabular}{|c|c|c|c|c|c|c|c|c|c|c|}
  \hline
  $\lambda_1$ & -1 & -1 & -1 & -1 & -1 & -1 & 1 & 1 & 1 & 2 \\
  $\lambda_2$ & -1 & -1 & -1 & 1  & 1  & 2  & 1 & 1  & 2  & 2 \\
  $\lambda_3$ & -1 & 1  & 2  & 1  & 2  & 2  & 1 & 2 & 2 & 2 \\
  \hline
\end{tabular}
\caption{$\lambda$ pairs}
\label{tab:basic_example}
\end{table}

\subsection*{2)}

\textbf{Solution.}

Since the eigenvalues of $A$ are:
$$\lambda_1 = 1, \quad \lambda_2 = -1, \quad \lambda_3 \in \{-1, 1, 2\}$$
we can get that the eigenvalues of $(A + I)$ are:
$$\lambda_1' = 1 + 1=2, \quad \lambda_2' = -1 + 1 =0, \quad \lambda_3' \in \{0, 2, 3\}$$

Thus,
$$
det(A + I) = \lambda_1' \cdot \lambda_2' \cdot \lambda_3' = 0
$$

\section*{Problem 2. (Eigenvector and similarity)}

\subsection*{1)}

\textbf{Solution.}

Let $v_3$ be the eigenvector of A with corresponding eigenvalue $\lambda_3$.

Then

\begin{align*}
Av_3 &= \lambda_3 v_3   \\
A(k_1v_1 + k_2v_2)&=\lambda_3(k_1v_1+k_2v_2)    \\
\lambda_1k_1v_1 + \lambda_2k_2v_2&=\lambda_3k_1v_1+\lambda_3k_2v_2) \\
k_1(\lambda_1 - \lambda_3)v_1 + k_2(\lambda_2 - \lambda_3)v_2 &= 0
\end{align*}

Since $\lambda_1, \lambda_2$ are distince eigenvalue,
we can get that $v_1,v_2$ are linearly independent.

Thus,

$$
k_1(\lambda_1-\lambda_3) = 0, \quad k_2(\lambda_2 - \lambda_3) = 0
$$

If $k_1 \neq 0$, then $\lambda_1 = \lambda_3$.
Since $\lambda_1 \neq \lambda_2$,
we can get that $\lambda_3 \neq \lambda_2$.
Thus $k_2$ = 0.

If $k_2 \neq 0$, then $\lambda_3 = \lambda_2$.
Since $\lambda_2 \neq \lambda_1$,
we can get that $\lambda_3 \neq \lambda_1$.
Thus $k_2$ = 0.

Above, $k_1k_2 = 0$.

\subsection*{2)}

\textbf{Solution.}

Two matrices are similar if and only if they have the same Jordan Normal Form.

The condition can only tell us the number of Jordan blocks associated with an eigenvalue,
and the sum of the sizes of those blocks.

\textbf{Counter Example:}

Define:

Matrix $\mathbf{A}$ (Partition 2+2):

$$\mathbf{A} =
\begin{pmatrix}
0 & 1 & 0 & 0 \\
0 & 0 & 0 & 0 \\
0 & 0 & 0 & 1 \\
0 & 0 & 0 & 0
\end{pmatrix}$$

Matrix $\mathbf{B}$ (Partition 3+1):

$$\mathbf{B} =
\begin{pmatrix}
0 & 1 & 0 & 0 \\
0 & 0 & 1 & 0 \\
0 & 0 & 0 & 0 \\
0 & 0 & 0 & 0
\end{pmatrix}$$

It is obviously that $\A$ and $\B$ satisfy the condition.

If $\A$ is similar with $\B$, we can easily get that $\A^2$ is similar with $\B^2$.

But

$$\mathbf{A}^2 = \mathbf{0}$$

$$\mathbf{B}^2 =
\begin{pmatrix} 0 & 0 & 1 & 0 \\ 0 & 0 & 0 & 0 \\ 0 & 0 & 0 & 0 \\ 0 & 0 & 0 & 0
\end{pmatrix} \neq \mathbf{0}$$

Thus $\A$ is not similar with $\B$.

\section*{Problem 3. (Diagonalization)}

\subsection*{1)}

\textbf{Solution.}

\textbf{Analysis of Matrix A}

\begin{align*}
\det(\lambda \mathbf{I} - \mathbf{A}) &= \det
\begin{bmatrix}\lambda + 2 & -1 & -1 \\ 0 & \lambda - 2 & 0 \\ 4 & -1 & \lambda -3
\end{bmatrix}   \\
&=(\lambda-2) \cdot \det
\begin{bmatrix} 2+\lambda & -1 \\ 4 & \lambda-3
\end{bmatrix} \\
&= (\lambda - 2)^2 (\lambda + 1)
\end{align*}

The eigenvalues are:
\begin{itemize}
\item $\lambda_1 = 2$ with algebraic multiplicity 2.
\item $\lambda_2 = -1$ with algebraic multiplicity 1.
\end{itemize}

Eigenspace for $\lambda_1 = 2$:

Fisrt we solve $(\mathbf{A} - 2\mathbf{I})\mathbf{v} = \mathbf{0}$.
\[
\begin{bmatrix} -4 & 1 & 1 \\ 0 & 0 & 0 \\ -4 & 1 & 1
\end{bmatrix}
\begin{bmatrix} x \\ y \\ z
\end{bmatrix} =
\begin{bmatrix} 0 \\ 0 \\ 0
\end{bmatrix}
\]

We have two free variables ($x$ and $y$).
\begin{itemize}
\item Setting $x=1, y=0 \implies z=4$, we get $\mathbf{v}_1 =
\begin{bmatrix} 1 \\ 0 \\ 4
\end{bmatrix}$.
\item Setting $x=0, y=1 \implies z=-1$, we get $\mathbf{v}_2 =
\begin{bmatrix} 0 \\ 1 \\ -1
\end{bmatrix}$.
\end{itemize}

Since we have 2 independent eigenvector, the geometry multiplicity is 2 and equal to algebraic multiplicity.

Eigenspace for $\lambda_2 = -1$:

Next, we solve $(\mathbf{A} + \mathbf{I})\mathbf{v} = \mathbf{0}$.
\[
\begin{bmatrix} -1 & 1 & 1 \\ 0 & 3 & 0 \\ -4 & 1 & 4
\end{bmatrix}
\begin{bmatrix} x \\ y \\ z
\end{bmatrix} =
\begin{bmatrix} 0 \\ 0 \\ 0
\end{bmatrix}
\]

Setting $x=1$, we get $\mathbf{v}_3 =
\begin{bmatrix} 1 \\ 0 \\ 1
\end{bmatrix}$.

Geometric multiplicity is 1 and equal to algebraic multiplicity.

Since geometric multiplicities equal algebraic multiplicities for all eigenvalues, \textbf{$\mathbf{A}$ is diagonalizable.}

\textbf{Analysis of Matrix B}

\begin{align*}
\det(\lambda \mathbf{I} - \mathbf{B}) &= \det
\begin{bmatrix}\lambda - 2 & -4 & 0 \\ 0 & \lambda - 2 & 0 \\ 0 & -5 & \lambda -2
\end{bmatrix}   \\
&= (\lambda - 2)^3
\end{align*}

Thus, eigenvalue $\lambda = 2$ has algebraic multiplicity 3.

Eigenspace for $\lambda = 2$:

We solve $(\mathbf{B} - 2\mathbf{I})\mathbf{v} = \mathbf{0}$.

\[
\begin{bmatrix} 0 & 4 & 0 \\ 0 & 0 & 0 \\ 0 & 5 & 0
\end{bmatrix}
\begin{bmatrix} x \\ y \\ z
\end{bmatrix} =
\begin{bmatrix} 0 \\ 0 \\ 0
\end{bmatrix}
\]

We can easily get the basis vectors: $\mathbf{u}_1 =
\begin{bmatrix} 1 \\ 0 \\ 0
\end{bmatrix}$ and $\mathbf{u}_2 =
\begin{bmatrix} 0 \\ 0 \\ 1
\end{bmatrix}$.
Thus, the geometric multiplicity is 2.

Since geometric multiplicity $\neq$ algebraic multiplicity, \textbf{$\mathbf{B}$ is NOT diagonalizable.}

\subsection*{2)}

Based on question 1), it is obviously that

$$
\mathbf{P} =
\begin{bmatrix} \mathbf{v}_1 & \mathbf{v}_2 & \mathbf{v}_3
\end{bmatrix} =
\begin{bmatrix} 1 & 0 & 1 \\ 0 & 1 & 0 \\ 4 & -1 & 1
\end{bmatrix}
$$
$$
\mathbf{D} =
\begin{bmatrix} 2 & 0 & 0 \\ 0 & 2 & 0 \\ 0 & 0 & -1
\end{bmatrix}
$$

and $\mathbf{A} = \mathbf{P}\mathbf{D}\mathbf{P}^{-1}$

\subsection*{3)}
Fisrt, we use Gaussian transformation to calculate $\mathbf{P}^{-1}$
and get
\[
\mathbf{P}^{-1} =
\begin{bmatrix} -1/3 & 1/3 & 1/3 \\ 0 & 1 & 0 \\ 4/3 & -1/3 & -1/3
\end{bmatrix}
\]

\begin{align*}
\A^{100} &= (\mathbf{P}\mathbf{D}\mathbf{P}^{-1})^100   \\
&= \mathbf{P}\mathbf{D}^{100}\mathbf{P}^{-1}  \\
&= \mathbf{P}\mathbf{D}^{100}\mathbf{P}^{-1} \\
&=
\begin{bmatrix} 1 & 0 & 1 \\ 0 & 1 & 0 \\ 4 & -1 & 1
\end{bmatrix}
\begin{bmatrix} 2^{100} & 0 & 0 \\ 0 & 2^{100} & 0 \\ 0 & 0 & 1
\end{bmatrix}
\begin{bmatrix} -1/3 & 1/3 & 1/3 \\ 0 & 1 & 0 \\ 4/3 & -1/3 & -1/3
\end{bmatrix} \\
&=
\begin{bmatrix}
\frac{4 - 2^{100}}{3} & \frac{2^{100}-1}{3} & \frac{2^{100}-1}{3} \\[6pt]
0 & 2^{100} & 0 \\[6pt]
\frac{4(1-2^{100})}{3} & \frac{2^{100}-1}{3} & \frac{4\cdot 2^{100}-1}{3}
\end{bmatrix}
\end{align*}

\section*{Problem 4. (Schur Decomposition)}

\subsection*{1)}

\textbf{Solution.}

First, we calculate the eigenvalue of A.

$$\det(\lambda I - A) = \lambda^3 - 6\lambda^2 + 11\lambda - 6 = (\lambda-1)(\lambda-2)(\lambda-3)$$

Then we can get that
\begin{align*}
\lambda_1 = 1   \\
\lambda_2 = 2   \\
\lambda_3 = 3
\end{align*}

Next, we solve $(A-\lambda I)v=0$ to get the eigenvector.

To $\lambda_1$, we get $v_1 = [1, 1, 1]^T$;
To $\lambda_2$, we get $v_2 = [1, 2, 4]^T$;
To $\lambda_3$, we get $v_3 = [1, 3, 9]^T$;

Since A have three distint eigenvalue, it have three linearly independent eigenvector.
Then we can apply Gram-Schmidt to eigenvectors to get U.

To $\mathbf{u}_1$:

$$
\begin{aligned}
r_{11} &= \|a_1\| = \sqrt{1^2 + 1^2 + 1^2} = \sqrt{3} \\
q_1 &= \frac{a_1}{r_{11}} = \frac{1}{\sqrt{3}}
\begin{bmatrix} 1 \\ 1 \\ 1
\end{bmatrix}
\end{aligned}$$

To $\mathbf{u}_2$:
$$r_{12} = \langle q_1, a_2 \rangle = \left( \frac{1}{\sqrt{3}}
\begin{bmatrix} 1 & 1 & 1
\end{bmatrix} \right)
\begin{bmatrix} 1 \\ 2 \\ 4
\end{bmatrix} = \frac{1}{\sqrt{3}}(1+2+4) = \frac{7}{\sqrt{3}}$$

$$w_2 = a_2 - r_{12} q_1 =
\begin{bmatrix} 1 \\ 2 \\ 4
\end{bmatrix} - \frac{7}{\sqrt{3}} \left( \frac{1}{\sqrt{3}}
\begin{bmatrix} 1 \\ 1 \\ 1
\end{bmatrix} \right)
=
\begin{bmatrix} 1 \\ 2 \\ 4
\end{bmatrix} -
\begin{bmatrix} 7/3 \\ 7/3 \\ 7/3
\end{bmatrix}
=
\begin{bmatrix} -4/3 \\ -1/3 \\ 5/3
\end{bmatrix}$$

$$
\begin{aligned}
r_{22} &= \|w_2\| = \sqrt{\left(-\frac{4}{3}\right)^2 + \left(-\frac{1}{3}\right)^2 + \left(\frac{5}{3}\right)^2} = \frac{1}{3}\sqrt{16+1+25} = \frac{\sqrt{42}}{3} \\
q_2 &= \frac{w_2}{r_{22}} = \frac{3}{\sqrt{42}} \left( \frac{1}{3}
\begin{bmatrix} -4 \\ -1 \\ 5
\end{bmatrix} \right) = \frac{1}{\sqrt{42}}
\begin{bmatrix} -4 \\ -1 \\ 5
\end{bmatrix}
\end{aligned}$$

To $\mathbf{u}_3$:

$$r_{13} = \langle q_1, v_3 \rangle = \frac{1}{\sqrt{3}} (1\cdot 1 + 1\cdot 3 + 1\cdot 9) = \frac{13}{\sqrt{3}}$$
$$r_{23} = \langle q_2, v_3 \rangle = \frac{1}{\sqrt{42}} (-4\cdot 1 + (-1)\cdot 3 + 5\cdot 9) = \frac{1}{\sqrt{42}} (-4 - 3 + 45) = \frac{38}{\sqrt{42}}$$
$$w_3 = v_3 - r_{13}q_1 - r_{23}q_2 =
\begin{bmatrix} 1 \\ 3 \\ 9
\end{bmatrix} - \frac{13}{3}
\begin{bmatrix} 1 \\ 1 \\ 1
\end{bmatrix} - \frac{38}{42}
\begin{bmatrix} -4 \\ -1 \\ 5
\end{bmatrix} = \frac{1}{7}
\begin{bmatrix} 2 \\ -3 \\ 1
\end{bmatrix}$$
$$r_{33} = \|w_3\| = \frac{1}{7} \sqrt{2^2 + (-3)^2 + 1^2} = \frac{1}{7} \sqrt{14} = \frac{\sqrt{14}}{7}$$
$$q_3 = \frac{w_3}{r_{33}} = \left( \frac{1}{7}
\begin{bmatrix} 2 \\ -3 \\ 1
\end{bmatrix} \right) \cdot \frac{7}{\sqrt{14}} = \frac{1}{\sqrt{14}}
\begin{bmatrix} 2 \\ -3 \\ 1
\end{bmatrix}$$

Above,
$$Q =
\begin{bmatrix} \frac{1}{\sqrt{3}} & \frac{-4}{\sqrt{42}} & \frac{2}{\sqrt{14}} \\
\frac{1}{\sqrt{3}} & \frac{-1}{\sqrt{42}} & \frac{-3}{\sqrt{14}} \\
\frac{1}{\sqrt{3}} & \frac{5}{\sqrt{42}} & \frac{1}{\sqrt{14}}
\end{bmatrix}$$
$$R =
\begin{bmatrix}
\sqrt{3} & \frac{7}{\sqrt{3}} & \frac{13}{\sqrt{3}} \\
0 & \frac{\sqrt{42}}{3} & \frac{38}{\sqrt{42}} \\
0 & 0 & \frac{\sqrt{14}}{7}
\end{bmatrix}$$

Thus,
$$
U = Q =
\begin{bmatrix} \frac{1}{\sqrt{3}} & \frac{-4}{\sqrt{42}} & \frac{2}{\sqrt{14}} \\
\frac{1}{\sqrt{3}} & \frac{-1}{\sqrt{42}} & \frac{-3}{\sqrt{14}} \\
\frac{1}{\sqrt{3}} & \frac{5}{\sqrt{42}} & \frac{1}{\sqrt{14}}
\end{bmatrix}
=
\begin{bmatrix} 0.5774 & -0.6172 & 0.5345 \\ 0.5774 & -0.1543 & -0.8018 \\ 0.5774 & 0.7715 & 0.2673
\end{bmatrix}
$$
$$
T = R \Lambda R^{-1} =
\begin{bmatrix}
1 & \frac{7}{\sqrt{14}} & \frac{49}{\sqrt{42}} \\
0 & 2 & \frac{19}{\sqrt{3}} \\
0 & 0 & 3
\end{bmatrix}
=
\begin{bmatrix}
1.0000 & 1.8708 & 7.5607 \\
0 & 2.0000 & 10.9697 \\
0 & 0 & 3.0000
\end{bmatrix}
$$

\subsection*{2)}

\textbf{Solution.}

Let $Q^H B Q = C$.
Then,
\begin{align*}
Q^H AB Q &= Q^H BA Q  \\
Q^H A Q Q^H B Q&=Q^H B Q Q^H A Q  \\
TC &= CT
\end{align*}

Since T is upper triangular matrix, use the hint and we can get that C is upper triangular matrix.

\section*{Problem 5. (Variational Characterizations)}

\textbf{Solution.}

\textbf{Section 1:}
Fisrt, we prove
$\|\mathbf{A}\|_2 = \max_{\|\mathbf{z}\|_2=1} \|\mathbf{A}\mathbf{z}\|_2$

According to the definition of $\|\mathbf{A}\|_2$:
$$
\|\mathbf{A}\|_2 = \sup_{\mathbf{x} \neq 0} \frac{\|\mathbf{A}\mathbf{x}\|_2}{\|\mathbf{x}\|_2}
$$

Let $\x = c \cdot \z$, and we can get that
\begin{align*}
\|\mathbf{A}\|_2 &= \sup_{\|\mathbf{z}\|_2=1} \frac{\|c \cdot \mathbf{A}\mathbf{z}\|_2}{\|c \cdot \mathbf{z}\|_2} \\
&= \sup_{\|\mathbf{z}\|_2=1} \frac{|c| \cdot \| \mathbf{A}\mathbf{z}\|_2}{|c| \cdot \|\mathbf{z}\|_2} \\
&= \sup_{\|\mathbf{z}\|_2=1} \frac{\| \mathbf{A}\mathbf{z}\|_2}{\|\mathbf{z}\|_2}
\end{align*}

\textbf{Section 2:}
Next, we prove $\max_{\|\mathbf{z}\|_2=1} \|\mathbf{A}\mathbf{z}\|_2 = \max_{\|\mathbf{z}\|_2=\|\mathbf{y}\|_2=1} |\mathbf{y}^T \mathbf{A}\mathbf{z}|$

We need to prove

$$\max_{\|\mathbf{z}\|_2=1} \|\mathbf{A}\mathbf{z}\|_2 \geq \max_{\|\mathbf{z}\|_2=\|\mathbf{y}\|_2=1} |\mathbf{y}^T \mathbf{A}\mathbf{z}|$$
and
$$\max_{\|\mathbf{z}\|_2=\|\mathbf{y}\|_2=1} |\mathbf{y}^T \mathbf{A}\mathbf{z}| \geq \max_{\|\mathbf{z}\|_2=1} \|\mathbf{A}\mathbf{z}\|_2$$

\textbf{step 1:}

To solve $\max_{\|\mathbf{z}\|_2=1} \|\mathbf{A}\mathbf{z}\|_2 \geq \max_{\|\mathbf{z}\|_2=\|\mathbf{y}\|_2=1} |\mathbf{y}^T \mathbf{A}\mathbf{z}|$:

\begin{align*}
\max_{\|\mathbf{z}\|_2=\|\mathbf{y}\|_2=1} |\mathbf{y}^T \mathbf{A}\mathbf{z}| &\leq \max_{\|\mathbf{z}\|_2=\|\mathbf{y}\|_2=1} \|\mathbf{y}\|_2 \|\mathbf{A}\mathbf{z}\|_2 \\
&= \max_{\|\mathbf{z}\|_2=1}\|\mathbf{A}\mathbf{z}\|_2
\end{align*}

Thus, it satisfy this condition.

\textbf{step 2:}

To solve $\max_{\|\mathbf{z}\|_2=\|\mathbf{y}\|_2=1} |\mathbf{y}^T \mathbf{A}\mathbf{z}| \geq \max_{\|\mathbf{z}\|_2=1} \|\mathbf{A}\mathbf{z}\|_2$:

Let $$\mathbf{y} = \frac{\mathbf{Az}}{\|\mathbf{Az}\|_2}$$

$$|\mathbf{y}^T \mathbf{Az}| = |\left(\frac{\mathbf{Az}}{\|\mathbf{Az}\|_2}\right)^T \mathbf{Az}| = |\frac{(\mathbf{Az})^T (\mathbf{Az})}{\|\mathbf{Az}\|_2}| = \|\mathbf{Az}\|_2$$

Since there exist a $\y$ to make $\max_{\|\mathbf{z}\|_2=\|\mathbf{y}\|_2=1} |\mathbf{y}^T \mathbf{A}\mathbf{z}| \geq \max_{\|\mathbf{z}\|_2=1} \|\mathbf{A}\mathbf{z}\|_2$, we can get that:
$$
\max_{\|\mathbf{z}\|_2=\|\mathbf{y}\|_2=1} |\mathbf{y}^T \mathbf{A}\mathbf{z}| \geq \max_{\|\mathbf{z}\|_2=1} \|\mathbf{A}\mathbf{z}\|_2
$$

\section*{Problem 6. (Power Iteration)}


\end{document}
